%===============================================================================
% Zweck:    KTR-Präsentation-Vorlage
% Erstellt: 15.04.2013
% Update:   04.07.2016
% Autor:    M.G.
%===============================================================================

\newcommand\ratio{169}
\documentclass[10pt,aspectratio=\ratio,
%draft,
%handout,
compress
]{beamer}

\newcommand\meta{./meta}
\input{\meta/config/commands}


\def\signed #1{{\leavevmode\unskip\nobreak\hfil\penalty50\hskip2em
  \hbox{}\nobreak\hfil(#1)%
  \parfillskip=0pt \finalhyphendemerits=0 \endgraf}}

\newsavebox\mybox
\newenvironment{aquote}[1]
  {\savebox\mybox{#1}\begin{fancyquotes}}
  {\signed{\usebox\mybox}\end{fancyquotes}}


\input{\meta/config/hyphenation}

\setbeamertemplate{caption}[numbered]
%\numberwithin{figure}{section}

\begin{document}
  %===============================================================================
  % Zum Kompilieren latexmk ausführen.
  %	Konfiguration in texmaker: Options -> Configure Texmaker -> Quick Build -> Select Latexmk + ViewPD
  %	Entsprechende Informationen in den config/metainfo verändern
  % Zur Auswahl der Sprache im folgenden Befehl
  % ngerman für deutsch eintragen, english für Englisch.
  %===============================================================================
\selectlanguage{english}
\ifnum\ratio<169
\frame{\titlepage}
\else
\frame[plain]{\titlepage}
\fi

%\AtBeginSection[]
%{
%  \frame<handout:0>
%  {
%    \frametitle{Outline}
%    \tableofcontents[currentsection,hideallsubsections]
%  }
%}

\AtBeginSubsection[]
{
  \frame<handout:0>
  {
    \frametitle{Outline}
    \tableofcontents[sectionstyle=show/hide,subsectionstyle=show/shaded/hide,subsubsectionstyle=hide]
  }
}

\AtBeginSubsubsection[]
{
  \frame<handout:0>
  {
    \frametitle{Outline}
    \tableofcontents[sectionstyle=show/hide,subsectionstyle=show/shaded/hide,subsubsectionstyle=show/shaded/hide]
  }
}

\newcommand<>{\highlighton}[1]{%
  \alt#2{\structure{#1}}{{#1}}
}

\newcommand{\icon}[1]{\pgfimage[height=1em]{#1}}

\section*{}
\phantomsection
\begin{frame}{Content}
\tableofcontents
\end{frame}
%%%%%%%%%%%%%%%%%%%%%%%%%%%%%%%%%%%%%%%%%
%%%%%%%%%% Content starts here %%%%%%%%%%
%%%%%%%%%%%%%%%%%%%%%%%%%%%%%%%%%%%%%%%%%

\section{Project Goal}
\begin{frame}{Project Goal}\framesubtitle{}
	Implement a Prototype for a MQTT-SN gateway runnable on Semi-Constrained devices.

\end{frame}

\section{Related Work}
\subsection{Messaging Standards}
\begin{frame}{Eclipse IoT Developer Survey 2017}\framesubtitle{}
\image{.9\textwidth}{EclipseMessagingStandard.png}{Eclipse IoT Developer Survey 2017 - Messaging Standards}{}
\end{frame}

%\subsubsection{HTTP}
\begin{frame}{HTPP}\framesubtitle{}
\begin{itemize}
\item client-server paradigm
\item addressable resource (URI)
\end{itemize}
\image{.5\textwidth}{HTTPRequestResponseBehaviour.png}{publish subscribe example}{img:logo}
\end{frame}

%\subsubsection{MQTT}
\begin{frame}{MQTT}\framesubtitle{}
	\begin{itemize}
		\item publish-subscribe
		\item data centrich approach
	\end{itemize}
	\image{.5\textwidth}{MQTTPubSub.png}{HTTP request response}{img:logo}
\end{frame}

%\subsubsection{MQTT QoS}
\begin{frame}{MQTT QoS}\framesubtitle{}
	\begin{itemize}
		\item QoS 0 (at most once)
		\item QoS 1 (at least once)
		\item QoS 2 (exactly once)
	\end{itemize}
	\image{.5\textwidth}{MQTTQoS.png}{MQTT QoS methods}{img:logo}
\end{frame}

%\subsubsection{MQTT-SN}
\begin{frame}{MQTT-SN}\framesubtitle{}
	\begin{itemize}
		\item version of MQTT
		\item not connection oriented (no TCP)
		\item supports short \& pre-defined topic names
		\item supports sleeping clients
		\item QoS -1 (QoS 0 publish without connect)
	\end{itemize}
\end{frame}


%\subsubsection{Comparing HTTP, MQTT, MQTT-SN}
\begin{frame}{Comparing HTTP, MQTT, MQTT-SN}\framesubtitle{}
	\begin{itemize}
		\item comparing minimal packet sizes in TCP/IP Model
		\item minimum valuable example:	HTTP GET, MQTT publish QoS 0, MQTT-SN publish QoS 0
		\item result: MQTT + TCP + WiFi = 82 bytes vs MQTT-SN + BLE 21 bytes
	\end{itemize}
	\image{.5\textwidth}{TCPIPModelLayerWithLayerBytes.png}{TCP/IP Model Layer with minimal protocol length}{img:logo}
\end{frame}


\subsection{IoT Environments}
\begin{frame}{IoT Environments}\framesubtitle{}
	\image{.5\textwidth}{DeviceTypes.png}{IoT Environment from small hardware to the cloud}{img:logo}
\end{frame}

\subsection{Device Types}
\begin{frame}{Device Types}\framesubtitle{}
\begin{itemize}
	\item Constrained Devices
	\begin{itemize}
	\item Limited resources (FLASH, CPU, RAM, Energy)
	\end{itemize}
	\item Semi-Constrained Devices
	\begin{itemize}
		\item Limited resource (FLASH, CPU, RAM, not Energy)
	\end{itemize}
	\item Unconstrained Devices
	\begin{itemize}
		\item Nealy unlimited resources (HDD, CPU, RAM)
	\end{itemize}
\end{itemize}
\end{frame}

\subsection{Target Hardware and Software Environment}
\begin{frame}{Target Hardware and Software Environment}\framesubtitle{}
	\image{.17\textwidth}{SDCardTestBoard.jpg}{NodeMCU v2 (ESP8266) + SDHC Card as Target Environment}{img:logo}
	\image{.3\textwidth}{ArduinoCommunityLogo.png}{Arduino as Software Environment}{img:logo}
\end{frame}


\subsection{MQTT-SN Architecture}
%\subsubsection{MQTT-SN Architecture}
\begin{frame}{MQTT-SN Architecture}\framesubtitle{}
\begin{itemize}
	\item three MQTT-SN components: client, gateway, forwarder
\end{itemize}
	\image{.5\textwidth}{MQTTSNArchitecture.png}{MQTT-SN Architecture}{img:logo}
\end{frame}
\begin{frame}{MQTT-SN Gateway Architecture}\framesubtitle{}
	\begin{itemize}
		\item two kind of gateways: transparent and aggregating
	\end{itemize}
	\image{.8\textwidth}{TransparendAndAggregatingGateways.png}{MQTT-SN gateway architecture}{img:logo}
\end{frame}


%\subsubsection{MQTT-SN Client Lify Cycle}
\begin{frame}{MQTT-SN Client Lify Cycle}\framesubtitle{}
	\begin{itemize}
		\item MQTT-SN client as a constrained device:
		\begin{itemize}
			\item find a MQTT-SN Gateway via advertisement or searching a gateway
			\item connect to the MQTT-SN gateway with a will message
			\item register topics
			\item subscribe to topics
			\item send and receive publishes
			\item unsubscribe from topics
			\item sleep 
			\item wake up and collect queued publishes
			\item sleep and wake up frequently
			\item power source is empty - will message is published
		\end{itemize}
			\end{itemize}
\end{frame}

\subsection{Transmission Protocols}
\begin{frame}{Transmission Protocols}\framesubtitle{}
	\image{.8\textwidth}{TransmissionProtocols.png}{example transmission protocols}{img:logo}
\end{frame}

\subsection{Overview - MQTT-SN, Transmission Protocols, Device Types}
\begin{frame}{Overviewv - MQTT-SN, Transmission Protocols, Device Types}\framesubtitle{}
	\image{.8\textwidth}{MQTTSNTransmissionProtocolsAndDeviceTypes.png}{MQTT-SN, Transmission Protocols, Device Types}{img:logo}
\end{frame}

\section{MQTT-SN Gateway Implementation}
\subsection{Core Components}
\begin{frame}{Core Components}\framesubtitle{}
			\begin{itemize}
				\item environment independent implementation
				\item MqttSnMessageHandler + Core implemented - rest interfaces
			\end{itemize}
	\image{.8\textwidth}{CoreClassDiagram.png}{Core Component class diagram}{img:logo}
\end{frame}

\subsection{Gatway Class}
\begin{frame}{Core Components}\framesubtitle{}
	\begin{itemize}
		\item holds references to implementation - single class to embed
		\item initializes components' references
		\item loop()s over components
	\end{itemize}
	\image{.5\textwidth}{GatewayClassCollaborationDiagram.png}{Gateway class collaboration diagram}{img:logo}
\end{frame}


\subsection{Linux Gateway Implementation}
\begin{frame}{Linux Gateway Implementation}\framesubtitle{}
	\image{.5\textwidth}{LinuxClassDiagramWithIcons.png}{Linux Gateway implements Core Component's interfce. Using UDP and Paho embedded C/C++ MQTT Client}{img:logo}
\end{frame}

\subsection{Unit and Regression Testing}
\begin{frame}{Unit and Regression Testing}\framesubtitle{}
	\begin{itemize}
		\item using GoogleTest and GoogleMock
		\item starting tests inside IDE CLion
		\item using Docker for running MQTT broker (Mosquitto)
		\item writing a MqttSnTestClient + PahoMqttTestMessageHandler
	\end{itemize}
		\image{.5\textwidth}{TestEnvironment.png}{Clion, GoogleTest, Docker, Mosquitto}{img:logo}
\end{frame}

\subsection{MQTT-SN \& MQTT Test Clients}
\begin{frame}{MQTT-SN \& MQTT Test Client}\framesubtitle{}
	\image{.7\textwidth}{MqttSnTestClientCollaborationDiagram.png}{MQTT-SN test client class diagram}{img:logo}
	\image{.7\textwidth}{MqttTestMessageHandlerClassDiagram.png}{MQTT test client class diagram}{img:logo}
\end{frame}


\subsection{Test Results}
\begin{frame}{Test Results}\framesubtitle{}
	\begin{itemize}
		\item compliance tests (well and ill-formed MQTT-SN packets)
		\item functional tests (behaviour correct)
		\item Total: 96 unit tests
		\item 90 pass + 6 fail (QoS 2 not implemented but tested)
		\item Not everyting is tested, but: Important functionality is tested and working
	\end{itemize}
\end{frame}

\subsection{BLESocket}
\begin{frame}{BLESocket}\framesubtitle{}
	\begin{itemize}
		\item drop in replacement for LinuxUdpSocket
		\item uses self written SimpleBluetoothLowEnergySocket
	\end{itemize}
		\image{.7\textwidth}{BLESocketCollaborationDiagram.png}{BLESocket class diagram}{img:logo}
\end{frame}

\section{Conclustion \& Future Work}
\begin{frame}{Conclustion}\framesubtitle{}
	\begin{itemize}
		\item tl;dr: project successful
		\item implemented a MQTT-SN gateway prototype on Linux
		\item runs partly tested on target hardware and software environment (ESP8266+SDCard)
		\item tested (reusable)
		\item designed to be easy adaptable to multiple environments
	\end{itemize}
\end{frame}

\begin{frame}{Future work}\framesubtitle{}
	\begin{itemize}
		\item implement a MQTT-SN client for constrained devices
		\item more transmission protocols (WS17/18 LoRa)
		\item support more platforms: Mbed, RTOS
		\item implement: QoS 2 \& will update
		\item enhance tests: more tests, stress tests, measure code coverage
	\end{itemize}
\end{frame}
%%%%%%%%%%%%%%%%%%%%%%%%%%%%%%%%%%%%%%%%%
%%%%%%%%%% References          %%%%%%%%%%
%%%%%%%%%%%%%%%%%%%%%%%%%%%%%%%%%%%%%%%%%
%\section*{}
%\begin{frame}[allowframebreaks]{References}
% \def\newblock{\hskip .11em plus .33em minus .07em}
% \scriptsize
% \bibliographystyle{IEEEtran}
% \bibliography{\meta/exampleLiterature/bib}
% \normalsize
%\end{frame}

%% Last frame
\frame{
  \vspace{2cm}
  {\huge Questions ?}

  \vspace{20mm}
  \nocite*

  \begin{flushright}
    Gabriel Nikol

    \structure{\footnotesize{\href{https://github.com/S3ler}{github.com/S3ler}}}
  \end{flushright}
}


\end{document}
